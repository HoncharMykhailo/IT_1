\documentclass[a4paper,12pt]{article}

% Підтримка української мови
\usepackage[utf8]{inputenc}
\usepackage[ukrainian]{babel}

% Основні пакети для форматування
\usepackage{graphicx}
\usepackage{geometry}
\usepackage{indentfirst}
\usepackage{amsmath}
\usepackage{amssymb}
\usepackage{titlesec}
\usepackage{setspace}
\usepackage{float}
\usepackage{caption}
\usepackage{fancyhdr}
\usepackage{xcolor}
\usepackage{listings}

% Поля сторінки
\geometry{left=25mm,right=15mm,top=20mm,bottom=20mm}
\setstretch{1.2}

% Заголовки
\titleformat{\section}{\normalfont\bfseries\large}{\thesection.}{1em}{}
\titleformat{\subsection}{\normalfont\bfseries\normalsize}{\thesubsection.}{1em}{}

% Оформлення колонтитулів
\pagestyle{fancy}
\fancyhf{}
\fancyhead[L]{Лабораторна робота №1}
\fancyfoot[C]{\thepage}

% --- Налаштування listings для C# ---
\lstdefinestyle{CSharp}{
	language=[Sharp]C,
	basicstyle=\ttfamily\small,
	keywordstyle=\color{blue}\bfseries,
	stringstyle=\color{brown},
	commentstyle=\color{gray},
	numbers=left,
	numberstyle=\tiny,
	stepnumber=1,
	numbersep=8pt,
	backgroundcolor=\color{gray!5},
	frame=single,
	tabsize=4,
	breaklines=true,
	breakatwhitespace=false,
	showstringspaces=false,
	captionpos=b
}

\begin{document}
	
	% --- Титульна сторінка ---
	\begin{titlepage}
		\centering
		\vspace*{3cm}
		
		{\LARGE \textbf{Лабораторна робота №1, етап 2}}\\[0.5cm]
		{\Large з дисципліни}\\[0.3cm]
		{\Large \textbf{«Інформаційні технології»}}\\[3cm]
		
		{\large Виконав:}\\[0.3cm]
		{\Large \textbf{Гончар Михайло, група ТК-41}}\\[2cm]
		
		{\Large \textbf{Київський нацiональний унiверситет iменi Тараса Шевченка}}\\[0.5cm]
		{\Large \textbf{Факультет комп’ютерних наук i кiбернетики}}\\[3cm]
		
		{\large Дата виконання:}\\[0.3cm]
		{\Large \textbf{03 жовтня 2025 р.}}
		
		\vfill
	\end{titlepage}
	
	% --- Початок нумерації сторінок ---
	\setcounter{page}{1}
	\pagenumbering{arabic}
	\newpage
	
	\section*{Мета роботи}
	Ознайомитися з принципами побудови систем управління табличними базами даних, створити частково реалізовану програмну систему для роботи з базами даних, таблицями та полями, реалізувати функції додавання, редагування, видалення, збереження та порівняння таблиць.
	
	\section*{Завдання варіанту}
	\begin{itemize}
		\item Підтримати типи даних: ціле, дійсне, символ, рядок, комплексне ціле, комплексне дійсне.
		\item Реалізувати функцію \textbf{різниця таблиць} — порівняння значень між двома таблицями за вибраними полями.
	\end{itemize}
	
	\section*{Хід виконання роботи}
	
	\subsection*{1. Розробка структури бази даних}
	Було створено базу даних із такими основними сутностями:
	\begin{itemize}
		\item \textbf{Dbase} — база даних;
		\item \textbf{Table} — таблиця;
		\item \textbf{Field} — поле таблиці;
		\item \textbf{Row} — рядок;
		\item \textbf{Value} — значення;
		\item \textbf{Type} — тип даних поля.
	\end{itemize}
	
	Всі зв’язки між таблицями реалізовано через ORM Entity Framework.
	
	\subsection*{2. Інтерфейс користувача}
	Графічний інтерфейс створено на базі \textbf{Windows Forms}.  
	Користувач має змогу:
	\begin{itemize}
		\item створювати нову базу даних;
		\item створювати таблиці в межах бази;
		\item додавати поля;
		\item додавати рядки та редагувати значення;
		\item видаляти бази, таблиці, поля, рядки;
		\item зберігати всі бази даних у JSON-файл;
		\item завантажувати дані з JSON-файлу;
		\item порівнювати дві таблиці (функція <<Різниця таблиць>>).
	\end{itemize}
	
	\begin{figure}[H]
		\centering
		\includegraphics[width=0.9\textwidth]{../screenshots/desktop_dbs.png}
		\caption{Головне вікно програми з переліком баз даних}
	\end{figure}
	
	\subsection*{3. Реалізація функціональності}
	
	\subsubsection*{3.1. Створення та видалення об’єктів}
	У коді реалізовано керування станами:
	\begin{itemize}
		\item 0 — робота з базами;
		\item 1 — робота з таблицями;
		\item 2 — робота з полями та рядками.
	\end{itemize}
	
	\subsubsection*{3.2. Валідація введених значень}
	Для перевірки введених користувачем значень реалізовано метод:
	\begin{verbatim}
		public bool ValidateValue(string val, int typeId)
	\end{verbatim}
	
	Перевірка комплексних типів:
	\begin{verbatim}
		case 5: return Regex.IsMatch(val, @"^[+-]?\d+([+-]\d+)i$"); // комплексне ціле
		case 6: return Regex.IsMatch(val, @"^[+-]?(\d+(\.\d+)?)([+-](\d+(\.\d+)?))i$");
	\end{verbatim}
	
	\subsubsection*{3.3. Збереження та завантаження баз даних}
	Кнопки \textbf{Save} і \textbf{Load} реалізують експорт та імпорт у форматі JSON:
	\begin{verbatim}
		File.WriteAllText("backup.json", json);
		File.ReadAllText("backup.json");
	\end{verbatim}
	
	\subsubsection*{3.4. Порівняння таблиць (різниця таблиць)}
	Функція викликається кнопкою \textbf{“Diff”}.  
	Відкривається нове вікно, де користувач обирає поле з поточної таблиці, іншу таблицю та поле для порівняння. Програма відображає обидві таблиці поруч і виконує аналіз значень.
	
	\begin{figure}[H]
		\centering
		\includegraphics[width=0.9\textwidth]{../screenshots/desktop_diff.png}
		\caption{Вікно “Table Difference” з двома таблицями для порівняння}
	\end{figure}
	
	\subsection*{4. Особливості реалізації}
	\begin{itemize}
		\item ORM: \textbf{Entity Framework Core};
		\item Формат резервного копіювання: \textbf{JSON (Newtonsoft.Json)};
		\item Інтерфейс: \textbf{Windows Forms (C\#)};
		\item Повна підтримка CRUD-операцій.
	\end{itemize}
	
	
	\section*{5. Тестування програми (Unit-тести)}
	
	Для перевірки коректності роботи функцій було створено набір модульних тестів із використанням бібліотеки \textbf{xUnit}. Тестування охоплює:
	\begin{itemize}
		\item перевірку валідності введених значень;
		\item додавання нових рядків і створення пов’язаних значень;
		\item пошук відмінностей між таблицями.
	\end{itemize}
	
	\subsection*{5.1. Тести валідації}
	Мета тесту — переконатися, що метод \texttt{ValidateValue} коректно обробляє різні типи даних.
	
	\begin{lstlisting}[style=CSharp,caption={ValidationTests}]
		[Theory]
		[InlineData("123", 1, true)]   // int
		[InlineData("abc", 1, false)]  // int
		[InlineData("12.5", 2, true)]  // double
		[InlineData("x", 3, true)]     // char
		[InlineData("xyz", 3, false)]  // char
		public void TestValidateValue(string val, int typeId, bool expected)
		{
			var form = new skbd.Form1();
			var result = form.ValidateValue(val, typeId);
			Assert.Equal(expected, result);
		}
	\end{lstlisting}
	
	\subsection*{5.2. Тести для сервісу рядків}
	\begin{lstlisting}[style=CSharp,caption={RowServiceTests}]
		[Fact]
		public void AddNewRow_ShouldCreateRowAndValues()
		{
			using var context = GetInMemoryContext();
			
			var table = new Table { Name = "TestTable" };
			context.Tables.Add(table);
			context.SaveChanges();
			
			context.Fields.Add(new Field { TableId = table.Id, Name = "Field1" });
			context.Fields.Add(new Field { TableId = table.Id, Name = "Field2" });
			context.SaveChanges();
			
			var service = new RowService(context);
			var row = service.AddNewRow(table.Id);
			
			Assert.NotNull(row);
			Assert.True(row.Id > 0);
			
			var values = context.Values.Where(v => v.RowId == row.Id).ToList();
			Assert.Equal(2, values.Count);
			Assert.All(values, v => Assert.Equal("", v.Val));
		}
	\end{lstlisting}
	
	\subsection*{5.3. Тести для порівняння таблиць}
	\begin{lstlisting}[style=CSharp,caption={TableDiffServiceTests}]
		[Fact]
		public void GetDifferentRows_ShouldReturnRowsWithoutMatches()
		{
			using var context = GetInMemoryContext();
			var service = new TableDiffService(context);
			
			var diff = service.GetDifferentRows(1, "Col1", 2, "ColX");
			
			Assert.Single(diff);         
			Assert.Equal(2, diff[0].Id);  
		}
	\end{lstlisting}
	
	\subsection*{5.4. Результати тестів}
	Усі тести були виконані успішно. Це підтверджує, що всі основні модулі системи працюють коректно.
	
	\begin{figure}[H]
		\centering
		\includegraphics[width=0.8\textwidth]{../screenshots/desktop_test.png}
		\caption{Результати проходження unit-тестів у xUnit}
	\end{figure}
	
	\section*{Результати роботи}
	\begin{itemize}
		\item Створено функціональну систему управління табличними базами даних;
		\item Реалізовано підтримку комплексних типів;
		\item Реалізовано операцію різниці таблиць;
		\item Додано збереження/відновлення у форматі JSON;
		\item Забезпечено зручний графічний інтерфейс.
	\end{itemize}
	
	\section*{Висновки}
	У ході виконання лабораторної роботи:
	\begin{itemize}
		\item засвоєно принципи побудови систем управління табличними базами даних;
		\item реалізовано програмну систему з використанням ORM Entity Framework;
		\item додано підтримку комплексних чисел;
		\item створено інструмент для порівняння таблиць;
		\item отримано практичні навички роботи з C\#, WinForms, EF, JSON.
	\end{itemize}
	
\end{document}
